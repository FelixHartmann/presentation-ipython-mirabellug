% /solutions/conference-talks/conference-ornate-20min.fr.tex, 22/02/2006 De Sousa

\documentclass{beamer}

\usepackage[utf8]{inputenc}
\usepackage[T1]{fontenc}
\usepackage{fourier}

\usepackage[frenchb]{babel}

% \usepackage{natbib}

\usepackage{graphicx}

% \usepackage{multimedia}
% \usepackage{movie15}

% Default fixed font does not support bold face
\DeclareFixedFont{\ttb}{T1}{txtt}{bx}{n}{12} % for bold
\DeclareFixedFont{\ttm}{T1}{txtt}{m}{n}{12}  % for normal

% Custom colors
\usepackage{color}
\definecolor{deepblue}{rgb}{0,0,0.5}
\definecolor{deepred}{rgb}{0.6,0,0}
\definecolor{deepgreen}{rgb}{0,0.5,0}

\usepackage{listings}

\lstset{
language=Python,
basicstyle=\ttm,
otherkeywords={self},             % Add keywords here
keywordstyle=\ttb\color{deepblue},
emph={MyClass,__init__},          % Custom highlighting
emphstyle=\ttb\color{deepred},    % Custom highlighting style
stringstyle=\color{deepgreen},
frame=tb,                         % Any extra options here
showstringspaces=false            % 
inputencoding=utf8,
extendedchars=true,
literate=%
            {é}{{\'{e}}}1
            {è}{{\`{e}}}1
            {ê}{{\^{e}}}1
            {ë}{{\¨{e}}}1
            {û}{{\^{u}}}1
            {ù}{{\`{u}}}1
            {â}{{\^{a}}}1
            {à}{{\`{a}}}1
            {î}{{\^{i}}}1
            {ç}{{\c{c}}}1
            {Ç}{{\c{C}}}1
            {É}{{\'{E}}}1
            {Ê}{{\^{E}}}1
            {À}{{\`{A}}}1
            {Â}{{\^{A}}}1
            {Î}{{\^{I}}}1
}

% include svg with LaTeX from Inkscape
\newcommand{\executeiffilenewer}[3]{%
 \ifnum\pdfstrcmp{\pdffilemoddate{#1}}%
 {\pdffilemoddate{#2}}>0%
 {\immediate\write18{#3}}\fi%
}
\newcommand{\includesvg}[1]{%
 \executeiffilenewer{#1.svg}{#1.pdf}%
 {inkscape -z -D --file=#1.svg %
 --export-pdf=#1.pdf --export-latex}%
 \input{#1.pdf_tex}%
}

\usepackage{xspace}
\usepackage{hyperref}
\usepackage{amsmath, amssymb, amsthm}

\mode<presentation> {
  \usetheme{Boadilla}
  \usecolortheme{seahorse}

  \setbeamercovered{transparent}
  % ou autre chose (il est également possible de supprimer cette ligne)
}


\logo{\includegraphics[height=1.5cm]{mirabellug.jpeg}}

\title[Python, IPython, et le reste]{Python et le Notebook IPython}

\subtitle{Toute la puissance de Python dans une interface Ouèbe}

\author[Félix Hartmann]{par Félix, pour le Mirabellug}

\institute[Mirabellug]
{
}

\date{7 juin 2013}

\subject{presentation IPython Mirabellug}
% Inséré uniquement dans la page d'information du fichier PDF.


% À supprimer si vous ne voulez pas que la table des matières apparaisse
% au début de chaque section :

% \AtBeginSection[] {
%   \begin{frame}<beamer>{Lignes directrices}
%     \tableofcontents[currentsection,currentsubsection]
%   \end{frame}
% }

% Si vous souhaitez recouvrir vos transparents un à un,
% utilisez la commande suivante (pour plus d'info, voir la page 74 du manuel
% d'utilisation de Beamer (version 3.06) par Till Tantau) :

% \beamerdefaultoverlayspecification{<+->}


\begin{document}


\begin{frame}
  \titlepage
\end{frame}

% \begin{frame}{Lignes directrices}
%   \tableofcontents
%   % Vous pouvez, si vous le souhaiter ajouter l'option [pausesections]
% \end{frame}

\begin{frame}{Python en quelques points}
	
	\begin{itemize}
		\item Première version publiée en 1991 par Guido van Rossum.
		\item Licence BSD.
		\item Langage généraliste.
		\item Paradigmes impératif et objet.
		\item Typage dynamique.
	\end{itemize}		
	
\end{frame}


\begin{frame}[fragile]{L'indentation syntaxique}

En Python, en bloc n'est pas délimité par des crochets \{ et \}. C'est 
l'indentation qui sert de délimiteur.

\begin{lstlisting}
i = 10

while i > 0:
    print("Tic tac")
    i -= 1

print("BOUM !")
\end{lstlisting}
	
\end{frame}


\begin{frame}[fragile]{Les boucles for}

En Python, une boucle for est une itération sur un objet itérable (une liste, 
une chaine de caractères, un tableau\dots).

\medskip

Par exemple, si on veut faire une boucle sur les chiffres de 0 à 9, on crée 
une liste qui contient ces chiffres et on itère dessus:

\begin{lstlisting}
for i in range(10):
    if i < 2:
        print(str(i) + u" kilomètre à pied")
    else:
        print(str(i) + u" kilomètres à pied")
\end{lstlisting}

Le \texttt{u} devant les chaines de caractères sert à préciser que ce sont des 
chaines Unicode (par défaut dans la branche 3 de Python).

\bigskip 
	
\end{frame}


\begin{frame}{IPython en quelques points}

	\begin{itemize}
		\item Terminal interactif pour Python créé par Fernando Pérez.
		\item IPython Notebook (depuis la version 0.13): une interface Web riche 
			  pour IPython. La communication avec le navigateur est réalisée par
			  ZeroMQ.
	\end{itemize}

\end{frame}


\begin{frame}{Notebook}
	
	\begin{itemize}
		\item Un projet = un répertoire.
		\item Les Notebooks sont des fichiers JSON (.ipynb).
		\item Possibilité d'export en script Python simple (les autrex formats 
			  d'export, comme le HTML, sont plus expérimentaux).
		\item Un Notebook est composé de cellules, chaque cellule pouvant contenir
			  soit du code, soit du texte au format Markdown.
		\item Lancement: commande \texttt{ipython notebook} depuis le répertoire du 
		      projet.
	\end{itemize}
	
\end{frame}
	
\end{document}